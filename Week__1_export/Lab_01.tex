% Options for packages loaded elsewhere
\PassOptionsToPackage{unicode}{hyperref}
\PassOptionsToPackage{hyphens}{url}
%
\documentclass[
]{article}
\usepackage{lmodern}
\usepackage{amssymb,amsmath}
\usepackage{ifxetex,ifluatex}
\ifnum 0\ifxetex 1\fi\ifluatex 1\fi=0 % if pdftex
  \usepackage[T1]{fontenc}
  \usepackage[utf8]{inputenc}
  \usepackage{textcomp} % provide euro and other symbols
\else % if luatex or xetex
  \usepackage{unicode-math}
  \defaultfontfeatures{Scale=MatchLowercase}
  \defaultfontfeatures[\rmfamily]{Ligatures=TeX,Scale=1}
\fi
% Use upquote if available, for straight quotes in verbatim environments
\IfFileExists{upquote.sty}{\usepackage{upquote}}{}
\IfFileExists{microtype.sty}{% use microtype if available
  \usepackage[]{microtype}
  \UseMicrotypeSet[protrusion]{basicmath} % disable protrusion for tt fonts
}{}
\makeatletter
\@ifundefined{KOMAClassName}{% if non-KOMA class
  \IfFileExists{parskip.sty}{%
    \usepackage{parskip}
  }{% else
    \setlength{\parindent}{0pt}
    \setlength{\parskip}{6pt plus 2pt minus 1pt}}
}{% if KOMA class
  \KOMAoptions{parskip=half}}
\makeatother
\usepackage{xcolor}
\IfFileExists{xurl.sty}{\usepackage{xurl}}{} % add URL line breaks if available
\IfFileExists{bookmark.sty}{\usepackage{bookmark}}{\usepackage{hyperref}}
\hypersetup{
  pdftitle={Lab: Week 1},
  pdfauthor={36-350 -- Statistical Computing},
  hidelinks,
  pdfcreator={LaTeX via pandoc}}
\urlstyle{same} % disable monospaced font for URLs
\usepackage[margin=1in]{geometry}
\usepackage{color}
\usepackage{fancyvrb}
\newcommand{\VerbBar}{|}
\newcommand{\VERB}{\Verb[commandchars=\\\{\}]}
\DefineVerbatimEnvironment{Highlighting}{Verbatim}{commandchars=\\\{\}}
% Add ',fontsize=\small' for more characters per line
\usepackage{framed}
\definecolor{shadecolor}{RGB}{248,248,248}
\newenvironment{Shaded}{\begin{snugshade}}{\end{snugshade}}
\newcommand{\AlertTok}[1]{\textcolor[rgb]{0.94,0.16,0.16}{#1}}
\newcommand{\AnnotationTok}[1]{\textcolor[rgb]{0.56,0.35,0.01}{\textbf{\textit{#1}}}}
\newcommand{\AttributeTok}[1]{\textcolor[rgb]{0.77,0.63,0.00}{#1}}
\newcommand{\BaseNTok}[1]{\textcolor[rgb]{0.00,0.00,0.81}{#1}}
\newcommand{\BuiltInTok}[1]{#1}
\newcommand{\CharTok}[1]{\textcolor[rgb]{0.31,0.60,0.02}{#1}}
\newcommand{\CommentTok}[1]{\textcolor[rgb]{0.56,0.35,0.01}{\textit{#1}}}
\newcommand{\CommentVarTok}[1]{\textcolor[rgb]{0.56,0.35,0.01}{\textbf{\textit{#1}}}}
\newcommand{\ConstantTok}[1]{\textcolor[rgb]{0.00,0.00,0.00}{#1}}
\newcommand{\ControlFlowTok}[1]{\textcolor[rgb]{0.13,0.29,0.53}{\textbf{#1}}}
\newcommand{\DataTypeTok}[1]{\textcolor[rgb]{0.13,0.29,0.53}{#1}}
\newcommand{\DecValTok}[1]{\textcolor[rgb]{0.00,0.00,0.81}{#1}}
\newcommand{\DocumentationTok}[1]{\textcolor[rgb]{0.56,0.35,0.01}{\textbf{\textit{#1}}}}
\newcommand{\ErrorTok}[1]{\textcolor[rgb]{0.64,0.00,0.00}{\textbf{#1}}}
\newcommand{\ExtensionTok}[1]{#1}
\newcommand{\FloatTok}[1]{\textcolor[rgb]{0.00,0.00,0.81}{#1}}
\newcommand{\FunctionTok}[1]{\textcolor[rgb]{0.00,0.00,0.00}{#1}}
\newcommand{\ImportTok}[1]{#1}
\newcommand{\InformationTok}[1]{\textcolor[rgb]{0.56,0.35,0.01}{\textbf{\textit{#1}}}}
\newcommand{\KeywordTok}[1]{\textcolor[rgb]{0.13,0.29,0.53}{\textbf{#1}}}
\newcommand{\NormalTok}[1]{#1}
\newcommand{\OperatorTok}[1]{\textcolor[rgb]{0.81,0.36,0.00}{\textbf{#1}}}
\newcommand{\OtherTok}[1]{\textcolor[rgb]{0.56,0.35,0.01}{#1}}
\newcommand{\PreprocessorTok}[1]{\textcolor[rgb]{0.56,0.35,0.01}{\textit{#1}}}
\newcommand{\RegionMarkerTok}[1]{#1}
\newcommand{\SpecialCharTok}[1]{\textcolor[rgb]{0.00,0.00,0.00}{#1}}
\newcommand{\SpecialStringTok}[1]{\textcolor[rgb]{0.31,0.60,0.02}{#1}}
\newcommand{\StringTok}[1]{\textcolor[rgb]{0.31,0.60,0.02}{#1}}
\newcommand{\VariableTok}[1]{\textcolor[rgb]{0.00,0.00,0.00}{#1}}
\newcommand{\VerbatimStringTok}[1]{\textcolor[rgb]{0.31,0.60,0.02}{#1}}
\newcommand{\WarningTok}[1]{\textcolor[rgb]{0.56,0.35,0.01}{\textbf{\textit{#1}}}}
\usepackage{longtable,booktabs}
% Correct order of tables after \paragraph or \subparagraph
\usepackage{etoolbox}
\makeatletter
\patchcmd\longtable{\par}{\if@noskipsec\mbox{}\fi\par}{}{}
\makeatother
% Allow footnotes in longtable head/foot
\IfFileExists{footnotehyper.sty}{\usepackage{footnotehyper}}{\usepackage{footnote}}
\makesavenoteenv{longtable}
\usepackage{graphicx,grffile}
\makeatletter
\def\maxwidth{\ifdim\Gin@nat@width>\linewidth\linewidth\else\Gin@nat@width\fi}
\def\maxheight{\ifdim\Gin@nat@height>\textheight\textheight\else\Gin@nat@height\fi}
\makeatother
% Scale images if necessary, so that they will not overflow the page
% margins by default, and it is still possible to overwrite the defaults
% using explicit options in \includegraphics[width, height, ...]{}
\setkeys{Gin}{width=\maxwidth,height=\maxheight,keepaspectratio}
% Set default figure placement to htbp
\makeatletter
\def\fps@figure{htbp}
\makeatother
\setlength{\emergencystretch}{3em} % prevent overfull lines
\providecommand{\tightlist}{%
  \setlength{\itemsep}{0pt}\setlength{\parskip}{0pt}}
\setcounter{secnumdepth}{-\maxdimen} % remove section numbering

\title{Lab: Week 1}
\author{36-350 -- Statistical Computing}
\date{Week 1 -- Fall 2020}

\begin{document}
\maketitle

Name:

Andrew ID:

You must submit \textbf{your own} lab as a PDF file on Gradescope.

\begin{center}\rule{0.5\linewidth}{0.5pt}\end{center}

After each question, you will see the following:

\begin{Shaded}
\begin{Highlighting}[]
\CommentTok{# FILL ME IN}
\end{Highlighting}
\end{Shaded}

This, in \texttt{R\ Markdown} parlance, is a ``code chunk.'' To answer
the question, replace this line with your answer. Note that anything
following a ``\#'' symbol is a comment (or is code that is ``commented
out''). Also note that you do not need to remove the question or make
other edits. Just fill in the code chunks.

To run the chunk to see if it works, simply put your cursor
\emph{inside} the chunk and, e.g., select ``Run Current Chunk'' from the
``Run'' pulldown tab. Alternately, you can click on the green arrow at
the upper right-most part of the chunk, or use
``\textless cntl\textgreater-\textless return\textgreater{}'' as a
keyboard shortcut.

Here is an example (that only makes sense if you are looking at the Rmd
file):

\hypertarget{question-0}{%
\subsection{Question 0}\label{question-0}}

Print ``Hello, world.''

\begin{Shaded}
\begin{Highlighting}[]
\KeywordTok{print}\NormalTok{(}\StringTok{"Hello, world."}\NormalTok{)}
\end{Highlighting}
\end{Shaded}

\begin{verbatim}
## [1] "Hello, world."
\end{verbatim}

For some questions, you will also be prompted for, e.g., written
explanations. For these, in addition to a code chunk, you will also see
the following:

\begin{verbatim}
FILL ME IN
\end{verbatim}

Note how there is no ``\{r linewidth=80\}'' following the tick marks in
the first line. This is a verbatim block; any text you write in this
block appears verbatim when you knit your file.

When you have finished answering the questions, click on the ``Knit''
button. This will output an HTML file; if you cannot find that file, go
to the console and type \texttt{getwd()} (i.e., ``get working
directory'')\ldots you may find that your working directory and the
directory in which you've placed the Rmd file are not the same. The HTML
file should be in the working directory.

\texttt{R\ Markdown} may prompt you to install packages to get the
knitting to work; do install these.

\begin{center}\rule{0.5\linewidth}{0.5pt}\end{center}

\hypertarget{vector-basics}{%
\section{Vector Basics}\label{vector-basics}}

\hypertarget{question-1}{%
\subsection{Question 1}\label{question-1}}

\emph{(3 points)}

\emph{Notes 1B (3)}

Initialize a vector \(y\) with one logical value, one numeric value, and
one character value, and determine the type of \(y\).

\begin{Shaded}
\begin{Highlighting}[]
\NormalTok{y =}\StringTok{ }\KeywordTok{c}\NormalTok{(}\OtherTok{FALSE}\NormalTok{,}\DecValTok{1}\NormalTok{,}\StringTok{"a"}\NormalTok{)}
\KeywordTok{typeof}\NormalTok{(y)}
\end{Highlighting}
\end{Shaded}

\begin{verbatim}
## [1] "character"
\end{verbatim}

\begin{verbatim}
"character"
\end{verbatim}

\hypertarget{question-2}{%
\subsection{Question 2}\label{question-2}}

\emph{(3 points)}

\emph{Notes 1B (5)}

Sort the vector \(y\) into ascending order. Comment on the order: what
type of ordering is it?

\begin{Shaded}
\begin{Highlighting}[]
\KeywordTok{sort}\NormalTok{(y)}
\end{Highlighting}
\end{Shaded}

\begin{verbatim}
## [1] "1"     "a"     "FALSE"
\end{verbatim}

\begin{verbatim}
alphabetical
\end{verbatim}

\hypertarget{question-3}{%
\subsection{Question 3}\label{question-3}}

\emph{(3 points)}

\emph{Notes 1B (3)}

Initialize a vector \(y\) of integers, with first value 4 and last value
-4, stepping down by 1. Do this \emph{two} different ways. After each
initialization, print the vector.

\begin{Shaded}
\begin{Highlighting}[]
\NormalTok{y =}\StringTok{ }\KeywordTok{seq}\NormalTok{(}\DecValTok{4}\NormalTok{,}\OperatorTok{-}\DecValTok{4}\NormalTok{,}\DataTypeTok{by=}\OperatorTok{-}\DecValTok{1}\NormalTok{)}
\NormalTok{y}
\end{Highlighting}
\end{Shaded}

\begin{verbatim}
## [1]  4  3  2  1  0 -1 -2 -3 -4
\end{verbatim}

\begin{Shaded}
\begin{Highlighting}[]
\NormalTok{y =}\StringTok{ }\DecValTok{4}\OperatorTok{:-}\DecValTok{4}
\NormalTok{y}
\end{Highlighting}
\end{Shaded}

\begin{verbatim}
## [1]  4  3  2  1  0 -1 -2 -3 -4
\end{verbatim}

\begin{center}\rule{0.5\linewidth}{0.5pt}\end{center}

Mathematical operations between vectors was not covered directly in
class. Standard operations include

\begin{longtable}[]{@{}ll@{}}
\toprule
Operation & Description\tabularnewline
\midrule
\endhead
+ & addition\tabularnewline
- & subtraction\tabularnewline
* & multiplication\tabularnewline
/ & division\tabularnewline
\^{} & exponentiation\tabularnewline
\%\% & modulus (i.e., remainder)\tabularnewline
\%/\% & division with (floored) integer round-off\tabularnewline
\bottomrule
\end{longtable}

Note the concept of vectorization: if \(x\) is an \(n\)-element vector,
and \(y\) is an \(n\)-element vector, then, e.g., \(x+y\) is an
\(n\)-element vector that contains the sums of the first elements and of
the second elements, etc. In other words, one does not have to loop over
vector indices to apply operations to each element.

\begin{center}\rule{0.5\linewidth}{0.5pt}\end{center}

\hypertarget{question-4}{%
\subsection{Question 4}\label{question-4}}

\emph{(3 points)}

\emph{Notes 1B (4)}

What variable type is 1? Divide 1 by 2. Note to yourself whether you get
zero or 0.5.

\begin{Shaded}
\begin{Highlighting}[]
\KeywordTok{typeof}\NormalTok{(}\DecValTok{1}\NormalTok{)}
\end{Highlighting}
\end{Shaded}

\begin{verbatim}
## [1] "double"
\end{verbatim}

\begin{Shaded}
\begin{Highlighting}[]
\DecValTok{1}\OperatorTok{/}\DecValTok{2}
\end{Highlighting}
\end{Shaded}

\begin{verbatim}
## [1] 0.5
\end{verbatim}

\hypertarget{question-5}{%
\subsection{Question 5}\label{question-5}}

\emph{(3 points)}

\emph{Use R Help Pane or Google}

Apply the \texttt{append()}, \texttt{cbind()}, and \texttt{rbind()}
functions to the two vectors defined below. Note that ``cbind'' means
``column bind'' (glue two columns together) and ``rbind'' means ``row
bind'' (glue two rows together). What is the class of the output from
\texttt{cbind()} and \texttt{rbind()} functions?

\begin{Shaded}
\begin{Highlighting}[]
\NormalTok{x =}\StringTok{ }\DecValTok{7}\OperatorTok{:}\DecValTok{9}
\NormalTok{y =}\StringTok{ }\DecValTok{4}\OperatorTok{:}\DecValTok{6}
\KeywordTok{append}\NormalTok{(x,y)}
\end{Highlighting}
\end{Shaded}

\begin{verbatim}
## [1] 7 8 9 4 5 6
\end{verbatim}

\begin{Shaded}
\begin{Highlighting}[]
\KeywordTok{cbind}\NormalTok{(x,y)}
\end{Highlighting}
\end{Shaded}

\begin{verbatim}
##      x y
## [1,] 7 4
## [2,] 8 5
## [3,] 9 6
\end{verbatim}

\begin{Shaded}
\begin{Highlighting}[]
\KeywordTok{rbind}\NormalTok{(x,y)}
\end{Highlighting}
\end{Shaded}

\begin{verbatim}
##   [,1] [,2] [,3]
## x    7    8    9
## y    4    5    6
\end{verbatim}

\begin{verbatim}
data frame
\end{verbatim}

\hypertarget{question-6}{%
\subsection{Question 6}\label{question-6}}

\emph{(3 points)}

\emph{Notes 1B (5)}

Use the \texttt{append()} and \texttt{rev()} functions to merge the
vectors \(x\) and \(y\) such that the output is 9, 4, 5, 6, 8, 7.

\begin{Shaded}
\begin{Highlighting}[]
\KeywordTok{append}\NormalTok{(}\KeywordTok{append}\NormalTok{(}\KeywordTok{rev}\NormalTok{(x)[[}\DecValTok{1}\NormalTok{]],y),}\KeywordTok{rev}\NormalTok{(x[}\DecValTok{1}\OperatorTok{:}\DecValTok{2}\NormalTok{]))}
\end{Highlighting}
\end{Shaded}

\begin{verbatim}
## [1] 9 4 5 6 8 7
\end{verbatim}

\hypertarget{logical-filtering}{%
\section{Logical Filtering}\label{logical-filtering}}

\hypertarget{question-7}{%
\subsection{Question 7}\label{question-7}}

\emph{(3 points)}

\emph{Notes 1B (6-7)}

Take the vector \(x\) defined below and display the elements that are
less than \(-1\) or greater than \(1\). Do this using the logical or
symbol, and again via the use of the \texttt{abs()} function (for
absolute value).

\begin{Shaded}
\begin{Highlighting}[]
\KeywordTok{set.seed}\NormalTok{(}\DecValTok{199}\NormalTok{)}
\NormalTok{x =}\StringTok{ }\KeywordTok{rnorm}\NormalTok{(}\DecValTok{20}\NormalTok{)}
\NormalTok{x[}\KeywordTok{abs}\NormalTok{(x)}\OperatorTok{>}\DecValTok{1} \OperatorTok{|}\StringTok{ }\NormalTok{x }\OperatorTok{>}\DecValTok{1}\NormalTok{]}
\end{Highlighting}
\end{Shaded}

\begin{verbatim}
## [1] -1.909143 -2.216337 -1.132455 -1.763385  1.291574
\end{verbatim}

\hypertarget{question-8}{%
\subsection{Question 8}\label{question-8}}

\emph{(3 points)}

\emph{Notes 1B (4,8)}

What proportion of values in the vector \(x\) are less than 0.5? Use
\texttt{sum()} and \texttt{length()} in your answer.

\begin{Shaded}
\begin{Highlighting}[]
\KeywordTok{sum}\NormalTok{(x }\OperatorTok{>}\StringTok{ }\FloatTok{.5}\NormalTok{)}\OperatorTok{/}\KeywordTok{length}\NormalTok{(x)}
\end{Highlighting}
\end{Shaded}

\begin{verbatim}
## [1] 0.5
\end{verbatim}

\hypertarget{question-9}{%
\subsection{Question 9}\label{question-9}}

\emph{(3 points)}

\emph{Notes 1B (8) and R Help Pane/Google}

Use \texttt{any()} to determine whether any element of the vector \(x\)
is less than -1. If the returned value is \texttt{TRUE}, determine which
elements of \(x\) are less than -1.

\begin{Shaded}
\begin{Highlighting}[]
\ControlFlowTok{if}\NormalTok{(}\KeywordTok{any}\NormalTok{(x }\OperatorTok{<}\StringTok{ }\DecValTok{-1}\NormalTok{)) }\KeywordTok{which}\NormalTok{(x }\OperatorTok{<}\StringTok{ }\DecValTok{-1}\NormalTok{)}
\end{Highlighting}
\end{Shaded}

\begin{verbatim}
## [1]  1  3 10 11
\end{verbatim}

\hypertarget{question-10}{%
\subsection{Question 10}\label{question-10}}

\emph{(3 points)}

\emph{Notes 1B (5)}

Sort all the values of \(x\) in \emph{decreasing} order. Do this two
different ways.

\begin{Shaded}
\begin{Highlighting}[]
\KeywordTok{rev}\NormalTok{(}\KeywordTok{sort}\NormalTok{(x))}
\end{Highlighting}
\end{Shaded}

\begin{verbatim}
##  [1]  1.29157391  0.98448947  0.95177053  0.80563530  0.75756594  0.73688424
##  [7]  0.58058560  0.56505908  0.55516667  0.54160719  0.49414548  0.27542518
## [13] -0.04973667 -0.06946347 -0.29888471 -0.58057099 -1.13245520 -1.76338525
## [19] -1.90914272 -2.21633653
\end{verbatim}

\begin{Shaded}
\begin{Highlighting}[]
\NormalTok{x[}\KeywordTok{rev}\NormalTok{(}\KeywordTok{order}\NormalTok{(x))]}
\end{Highlighting}
\end{Shaded}

\begin{verbatim}
##  [1]  1.29157391  0.98448947  0.95177053  0.80563530  0.75756594  0.73688424
##  [7]  0.58058560  0.56505908  0.55516667  0.54160719  0.49414548  0.27542518
## [13] -0.04973667 -0.06946347 -0.29888471 -0.58057099 -1.13245520 -1.76338525
## [19] -1.90914272 -2.21633653
\end{verbatim}

\hypertarget{question-11}{%
\subsection{Question 11}\label{question-11}}

\emph{(3 points)}

\emph{Notes 1B (8) and R Help Pane/Google}

Replace all positive values in the vector \(x\) with zero, using
\texttt{which()}. Confirm that all values in the new vector are
\(\leq 0\) using \texttt{all()}.

\begin{Shaded}
\begin{Highlighting}[]
\NormalTok{x <-}\StringTok{ }\KeywordTok{replace}\NormalTok{(x,}\KeywordTok{which}\NormalTok{(x}\OperatorTok{>}\DecValTok{0}\NormalTok{),}\DecValTok{0}\NormalTok{)}
\KeywordTok{all}\NormalTok{(x }\OperatorTok{<=}\StringTok{ }\DecValTok{0}\NormalTok{)}
\end{Highlighting}
\end{Shaded}

\begin{verbatim}
## [1] TRUE
\end{verbatim}

\hypertarget{lists}{%
\section{Lists}\label{lists}}

\hypertarget{question-12}{%
\subsection{Question 12}\label{question-12}}

\emph{(3 points)}

\emph{Notes 1C (2-3)}

Create an empty list \(x\). Then define its \emph{second} entry as the
vector 2:4. Then print the list. What value does the first entry default
to?

\begin{Shaded}
\begin{Highlighting}[]
\NormalTok{x =}\StringTok{ }\KeywordTok{list}\NormalTok{()}
\NormalTok{x[[}\DecValTok{2}\NormalTok{]] =}\StringTok{ }\DecValTok{2}\OperatorTok{:}\DecValTok{4}
\NormalTok{x}
\end{Highlighting}
\end{Shaded}

\begin{verbatim}
## [[1]]
## NULL
## 
## [[2]]
## [1] 2 3 4
\end{verbatim}

\begin{verbatim}
NULL
\end{verbatim}

\hypertarget{question-13}{%
\subsection{Question 13}\label{question-13}}

\emph{(3 points)}

\emph{Use R Help Pane or Google}

Use the \texttt{names()} function to rename the list entries to
\texttt{x} and \texttt{y}. Print \(x\) to ensure your changes took hold.

\begin{Shaded}
\begin{Highlighting}[]
\KeywordTok{names}\NormalTok{(x) =}\StringTok{ }\KeywordTok{c}\NormalTok{(}\StringTok{'x'}\NormalTok{,}\StringTok{'y'}\NormalTok{)}
\NormalTok{x}
\end{Highlighting}
\end{Shaded}

\begin{verbatim}
## $x
## NULL
## 
## $y
## [1] 2 3 4
\end{verbatim}

\hypertarget{question-14}{%
\subsection{Question 14}\label{question-14}}

\emph{(3 points)}

\emph{Use R Help Pane or Google}

Change the name of the first entry of the list \(x\) to \texttt{a}. Do
this by setting something equal to ``a'', i.e., \emph{not} by simply
repeating your answer to Q13. Hint: \texttt{names()} returns a vector,
and you know how to change the values associated with individual entries
in a vector.

\begin{Shaded}
\begin{Highlighting}[]
\KeywordTok{names}\NormalTok{(x)[}\DecValTok{1}\NormalTok{] =}\StringTok{ 'a'}
\NormalTok{x}
\end{Highlighting}
\end{Shaded}

\begin{verbatim}
## $a
## NULL
## 
## $y
## [1] 2 3 4
\end{verbatim}

\hypertarget{data-frames}{%
\section{Data Frames}\label{data-frames}}

\hypertarget{question-15}{%
\subsection{Question 15}\label{question-15}}

\emph{(3 points)}

\emph{Notes 1C (5) and R Help Pane/Google}

Create a data frame \texttt{df} that has columns \texttt{x} and
\texttt{y} and has three rows. Use the \texttt{nrow()}, \texttt{ncol()},
and \texttt{dim()} functions to display the number of rows, the number
of columns, and the dimensions of \texttt{df}. Let the first column
contain numbers, and the second column contain logical values.

\begin{Shaded}
\begin{Highlighting}[]
\NormalTok{df =}\StringTok{ }\KeywordTok{data.frame}\NormalTok{(}\DataTypeTok{x=}\DecValTok{1}\OperatorTok{:}\DecValTok{3}\NormalTok{,}\DataTypeTok{y=}\KeywordTok{c}\NormalTok{(T,F,T))}
\KeywordTok{ncol}\NormalTok{(df)}
\end{Highlighting}
\end{Shaded}

\begin{verbatim}
## [1] 2
\end{verbatim}

\begin{Shaded}
\begin{Highlighting}[]
\KeywordTok{nrow}\NormalTok{(df)}
\end{Highlighting}
\end{Shaded}

\begin{verbatim}
## [1] 3
\end{verbatim}

\hypertarget{question-16}{%
\subsection{Question 16}\label{question-16}}

\emph{(3 points)}

\emph{Notes 1C (3-4)}

Add columns to \texttt{df} using the dollar sign operator, using the
double bracket notation with number, and using the double bracket
notation with character name.

\begin{Shaded}
\begin{Highlighting}[]
\NormalTok{df}\OperatorTok{$}\NormalTok{z <-}\StringTok{ }\KeywordTok{c}\NormalTok{(}\StringTok{'a'}\NormalTok{,}\StringTok{'b'}\NormalTok{,}\StringTok{'c'}\NormalTok{)}
\NormalTok{df[[}\DecValTok{4}\NormalTok{]] <-}\KeywordTok{c}\NormalTok{(}\DecValTok{4}\NormalTok{,}\DecValTok{5}\NormalTok{,}\DecValTok{6}\NormalTok{)}
\NormalTok{df[[}\StringTok{'B'}\NormalTok{]] <-}\KeywordTok{c}\NormalTok{(T,T,T)}
\end{Highlighting}
\end{Shaded}

\hypertarget{question-17}{%
\subsection{Question 17}\label{question-17}}

\emph{(3 points)}

\emph{Use R Help Pane or Google}

Use \texttt{row.names()} to change the names of the rows of \texttt{df}
to ``1st'', ``2nd'', and ``3rd''.

\begin{Shaded}
\begin{Highlighting}[]
\KeywordTok{row.names}\NormalTok{(df) =}\StringTok{ }\KeywordTok{c}\NormalTok{(}\StringTok{'1st'}\NormalTok{,}\StringTok{'2nd'}\NormalTok{,}\StringTok{'3rd'}\NormalTok{)}
\end{Highlighting}
\end{Shaded}

\hypertarget{question-18}{%
\subsection{Question 18}\label{question-18}}

\emph{(3 points)}

\emph{Use Google}

Display the contents of the first row of \texttt{df} using the row
number and then using the row name. Note that you access the elements of
a two-dimensional object using
\texttt{{[}row\ number/name,column\ number/name{]}}.

\begin{Shaded}
\begin{Highlighting}[]
\NormalTok{df[}\DecValTok{1}\NormalTok{,}\DecValTok{1}\OperatorTok{:}\DecValTok{5}\NormalTok{]}
\end{Highlighting}
\end{Shaded}

\begin{verbatim}
##     x    y z V4    B
## 1st 1 TRUE a  4 TRUE
\end{verbatim}

\begin{Shaded}
\begin{Highlighting}[]
\NormalTok{df[}\StringTok{'1st'}\NormalTok{,}\DecValTok{1}\OperatorTok{:}\DecValTok{5}\NormalTok{]}
\end{Highlighting}
\end{Shaded}

\begin{verbatim}
##     x    y z V4    B
## 1st 1 TRUE a  4 TRUE
\end{verbatim}

\hypertarget{matrices}{%
\section{Matrices}\label{matrices}}

\hypertarget{question-19}{%
\subsection{Question 19}\label{question-19}}

\emph{(3 points)}

\emph{Notes 1C (6)}

Initialize a 2 x 2 matrix where all the matrix elements are 1. Display
the matrix.

\begin{Shaded}
\begin{Highlighting}[]
\NormalTok{m =}\StringTok{ }\KeywordTok{matrix}\NormalTok{(}\KeywordTok{c}\NormalTok{(}\DecValTok{1}\NormalTok{,}\DecValTok{1}\NormalTok{,}\DecValTok{1}\NormalTok{,}\DecValTok{1}\NormalTok{),}\DataTypeTok{nrow=}\DecValTok{2}\NormalTok{)}
\NormalTok{m}
\end{Highlighting}
\end{Shaded}

\begin{verbatim}
##      [,1] [,2]
## [1,]    1    1
## [2,]    1    1
\end{verbatim}

\hypertarget{question-20}{%
\subsection{Question 20}\label{question-20}}

\emph{(3 points)}

\emph{Notes 1C (6)}

Initialize another matrix that is 2 x 2, and fill the first column with
your first and last name, and the second column with the first and last
name of your favorite professor. (No pressure.) Display the matrix.

\begin{Shaded}
\begin{Highlighting}[]
\NormalTok{mw =}\StringTok{ }\KeywordTok{matrix}\NormalTok{(}\KeywordTok{c}\NormalTok{(}\StringTok{'Kyle'}\NormalTok{,}\StringTok{'Wagner'}\NormalTok{,}\StringTok{'Peter'}\NormalTok{,}\StringTok{'Freeman'}\NormalTok{),}\DataTypeTok{nrow=}\DecValTok{2}\NormalTok{)}
\NormalTok{mw}
\end{Highlighting}
\end{Shaded}

\begin{verbatim}
##      [,1]     [,2]     
## [1,] "Kyle"   "Peter"  
## [2,] "Wagner" "Freeman"
\end{verbatim}

\hypertarget{question-21}{%
\subsection{Question 21}\label{question-21}}

\emph{(3 points)}

\emph{Notes 1B (5)}

Flip the order of entries in the second column of the matrix in the last
question, in just one line of code. Display the matrix.

\begin{Shaded}
\begin{Highlighting}[]
\NormalTok{mw[}\DecValTok{1}\OperatorTok{:}\DecValTok{2}\NormalTok{,}\DecValTok{2}\NormalTok{] <-}\StringTok{ }\KeywordTok{rev}\NormalTok{(mw[}\DecValTok{1}\OperatorTok{:}\DecValTok{2}\NormalTok{,}\DecValTok{2}\NormalTok{])}
\NormalTok{mw}
\end{Highlighting}
\end{Shaded}

\begin{verbatim}
##      [,1]     [,2]     
## [1,] "Kyle"   "Freeman"
## [2,] "Wagner" "Peter"
\end{verbatim}

\hypertarget{question-22}{%
\subsection{Question 22}\label{question-22}}

\emph{(3 points)}

\emph{Notes 1C (6) and R Help Pane/Google}

Define a 2 x 2 matrix with elements 1, 2, 3, 4, and another with
elements 4, 3, 2, 1. Multiply the two using the \texttt{\%*\%} operator.
Then take the transpose of the second matrix and multiply the two
matrices. (See \texttt{t()}.) Then, last, compute the inverse of the
first matrix. (See \texttt{solve()}.) Verify that the matrix inverse
multiplies with the original matrix to yield the identity matrix.

\begin{Shaded}
\begin{Highlighting}[]
\NormalTok{a =}\StringTok{ }\KeywordTok{matrix}\NormalTok{(}\DecValTok{1}\OperatorTok{:}\DecValTok{4}\NormalTok{,}\DataTypeTok{nrow=}\DecValTok{2}\NormalTok{)}
\NormalTok{b =}\StringTok{ }\KeywordTok{matrix}\NormalTok{(}\DecValTok{4}\OperatorTok{:}\DecValTok{1}\NormalTok{,}\DataTypeTok{nrow=}\DecValTok{2}\NormalTok{)}
\NormalTok{a}\OperatorTok\NormalTok{b}
\end{Highlighting}
\end{Shaded}

\begin{verbatim}
##      [,1] [,2]
## [1,]   13    5
## [2,]   20    8
\end{verbatim}

\begin{Shaded}
\begin{Highlighting}[]
\NormalTok{a}\OperatorTok\KeywordTok{t}\NormalTok{(b)}
\end{Highlighting}
\end{Shaded}

\begin{verbatim}
##      [,1] [,2]
## [1,]   10    6
## [2,]   16   10
\end{verbatim}

\begin{Shaded}
\begin{Highlighting}[]
\NormalTok{inv <-}\StringTok{ }\KeywordTok{solve}\NormalTok{(a)}
\NormalTok{inv}\OperatorTok\NormalTok{a}
\end{Highlighting}
\end{Shaded}

\begin{verbatim}
##      [,1] [,2]
## [1,]    1    0
## [2,]    0    1
\end{verbatim}

\hypertarget{question-23}{%
\subsection{Question 23}\label{question-23}}

\emph{(3 points)}

\emph{Use Google}

When you define a (non-sparse) matrix, you set aside memory to hold the
contents of that matrix. Assuming that your matrix holds
double-precision floating-point numbers, and that your laptop's memory
is 8 GB, what is the largest square matrix (``square'' = same number of
rows and columns) that you can define? (An approximate answer is fine.)
This is an important consideration if, e.g., you have a set of \(n\)
data points and you wish to construct a matrix that contains all the
pairwise distances between points. If \(n\) gets too large, you will run
out of memory.

\begin{verbatim}
double is 8 bytes,8*10^9 bytes storage or 10 ^ 9 doubles, largest square matrix:10^4.5 rows, cols.
\end{verbatim}

\hypertarget{handy-vector-functions}{%
\section{Handy Vector Functions}\label{handy-vector-functions}}

Here we define some vectors:

\begin{Shaded}
\begin{Highlighting}[]
\KeywordTok{set.seed}\NormalTok{(}\DecValTok{1201}\NormalTok{)}
\NormalTok{u =}\StringTok{ }\KeywordTok{sample}\NormalTok{(}\DecValTok{100}\NormalTok{,}\DecValTok{100}\NormalTok{,}\DataTypeTok{replace=}\OtherTok{TRUE}\NormalTok{)}
\NormalTok{v =}\StringTok{ }\KeywordTok{sample}\NormalTok{(}\DecValTok{100}\NormalTok{,}\DecValTok{100}\NormalTok{,}\DataTypeTok{replace=}\OtherTok{TRUE}\NormalTok{)}
\NormalTok{l =}\StringTok{ }\KeywordTok{list}\NormalTok{(}\StringTok{"x"}\NormalTok{=}\KeywordTok{sample}\NormalTok{(}\DecValTok{1}\OperatorTok{:}\DecValTok{10}\NormalTok{,}\DecValTok{5}\NormalTok{),}\StringTok{"y"}\NormalTok{=}\KeywordTok{sample}\NormalTok{(}\DecValTok{11}\OperatorTok{:}\DecValTok{20}\NormalTok{,}\DecValTok{5}\NormalTok{))}
\NormalTok{df =}\StringTok{ }\KeywordTok{data.frame}\NormalTok{(}\StringTok{"x"}\NormalTok{=}\KeywordTok{sample}\NormalTok{(}\DecValTok{1}\OperatorTok{:}\DecValTok{10}\NormalTok{,}\DecValTok{5}\NormalTok{),}\StringTok{"y"}\NormalTok{=}\KeywordTok{sample}\NormalTok{(}\DecValTok{11}\OperatorTok{:}\DecValTok{20}\NormalTok{,}\DecValTok{5}\NormalTok{))}
\NormalTok{x =}\StringTok{ }\KeywordTok{c}\NormalTok{(}\DecValTok{1}\NormalTok{,}\DecValTok{2}\NormalTok{,}\DecValTok{3}\NormalTok{,}\DecValTok{4}\NormalTok{)}
\NormalTok{y =}\StringTok{ }\KeywordTok{c}\NormalTok{(}\OperatorTok{-}\DecValTok{2}\NormalTok{,}\DecValTok{2}\NormalTok{,}\OperatorTok{-}\DecValTok{3}\NormalTok{,}\DecValTok{3}\NormalTok{)}
\NormalTok{z =}\StringTok{ }\KeywordTok{c}\NormalTok{(}\OperatorTok{-}\DecValTok{5}\NormalTok{,}\DecValTok{1}\NormalTok{,}\DecValTok{2}\NormalTok{,}\OperatorTok{-}\DecValTok{4}\NormalTok{,}\DecValTok{3}\NormalTok{,}\DecValTok{4}\NormalTok{,}\OperatorTok{-}\DecValTok{3}\NormalTok{,}\DecValTok{6}\NormalTok{)}
\end{Highlighting}
\end{Shaded}

\hypertarget{question-24}{%
\subsection{Question 24}\label{question-24}}

\emph{(3 points)}

\emph{Notes 1D (2)}

Display the list \texttt{l} as a numerical vector, with names associated
with each element.

\begin{Shaded}
\begin{Highlighting}[]
\KeywordTok{unlist}\NormalTok{(l)}
\end{Highlighting}
\end{Shaded}

\begin{verbatim}
## x1 x2 x3 x4 x5 y1 y2 y3 y4 y5 
##  6  8  7  3  9 15 16 17 18 12
\end{verbatim}

\hypertarget{question-25}{%
\subsection{Question 25}\label{question-25}}

\emph{(3 points)}

\emph{Notes 1D (2)}

Display the list \texttt{l} as a numerical vector, while stripping away
the names seen in Q24.

\begin{Shaded}
\begin{Highlighting}[]
\KeywordTok{as.vector}\NormalTok{(}\KeywordTok{unlist}\NormalTok{(l))}
\end{Highlighting}
\end{Shaded}

\begin{verbatim}
##  [1]  6  8  7  3  9 15 16 17 18 12
\end{verbatim}

\hypertarget{question-26}{%
\subsection{Question 26}\label{question-26}}

\emph{(3 points)}

\emph{Notes 1B (5) and Notes 1D (2)}

Repeat Q25, but display the vector in \emph{descending} order.

\begin{Shaded}
\begin{Highlighting}[]
\KeywordTok{rev}\NormalTok{(}\KeywordTok{sort}\NormalTok{(}\KeywordTok{as.vector}\NormalTok{(}\KeywordTok{unlist}\NormalTok{(l))))}
\end{Highlighting}
\end{Shaded}

\begin{verbatim}
##  [1] 18 17 16 15 12  9  8  7  6  3
\end{verbatim}

\hypertarget{question-27}{%
\subsection{Question 27}\label{question-27}}

\emph{(3 points)}

\emph{Notes 1D (3)}

Here are the contents of the data frame \texttt{df}:

\begin{Shaded}
\begin{Highlighting}[]
\NormalTok{df}
\end{Highlighting}
\end{Shaded}

\begin{verbatim}
##   x  y
## 1 8 12
## 2 4 16
## 3 3 19
## 4 6 17
## 5 9 15
\end{verbatim}

Reorder the rows so that the entries of the \texttt{x} column are in
numerical order and the association between the ith entry of \texttt{x}
and the ith entry of \texttt{y} is not lost. Display the result.

\begin{Shaded}
\begin{Highlighting}[]
\NormalTok{df[}\KeywordTok{order}\NormalTok{(df[}\StringTok{'x'}\NormalTok{]),}\KeywordTok{c}\NormalTok{(}\StringTok{'x'}\NormalTok{,}\StringTok{'y'}\NormalTok{)]}
\end{Highlighting}
\end{Shaded}

\begin{verbatim}
##   x  y
## 3 3 19
## 2 4 16
## 4 6 17
## 1 8 12
## 5 9 15
\end{verbatim}

\hypertarget{question-28}{%
\subsection{Question 28}\label{question-28}}

\emph{(3 points)}

\emph{Notes 1B (4) and Notes 1D (4)}

Display the proportion of the total number of unique values in
\texttt{u} to the number of values in \texttt{u}.

\begin{Shaded}
\begin{Highlighting}[]
\KeywordTok{length}\NormalTok{(}\KeywordTok{unique}\NormalTok{(u))}\OperatorTok{/}\KeywordTok{length}\NormalTok{(u)}
\end{Highlighting}
\end{Shaded}

\begin{verbatim}
## [1] 0.62
\end{verbatim}

\hypertarget{question-29}{%
\subsection{Question 29}\label{question-29}}

\emph{(3 points)}

\emph{Notes 1D (5)}

Display a table that shows how often each value of \texttt{v} appears.

\begin{Shaded}
\begin{Highlighting}[]
\KeywordTok{table}\NormalTok{(v)}
\end{Highlighting}
\end{Shaded}

\begin{verbatim}
## v
## 2 4 5 7 11 12 13 14 15 16 17 22 23 26 27 28 29 30 33 35 36 37 40 41 42 43
## 5 1 1 1 2 1 3 1 3 1 2 1 1 3 5 1 1 2 3 1 1 1 1 1 5 1
## 45 46 47 49 50 52 54 57 58 61 62 64 65 66 67 73 74 77 78 79 80 81 84 87 88
92
## 1 2 1 3 2 2 1 3 1 1 1 2 2 1 3 2 2 1 1 2 2 1 3 1 1 2
## 93 94 95 96 98
## 1 1 2 1 2
\end{verbatim}

\hypertarget{question-30}{%
\subsection{Question 30}\label{question-30}}

\emph{(3 points)}

\emph{Notes 1D (7)}

How many unique values do \texttt{u} and \texttt{v} have in common?

\begin{Shaded}
\begin{Highlighting}[]
\KeywordTok{length}\NormalTok{(}\KeywordTok{unique}\NormalTok{(}\KeywordTok{intersect}\NormalTok{(u,v)))}
\end{Highlighting}
\end{Shaded}

\begin{verbatim}
## [1] 39
\end{verbatim}

\hypertarget{question-31}{%
\subsection{Question 31}\label{question-31}}

\emph{(3 points)}

\emph{Notes 1B (4) and Notes 1D (7)}

Write down an expression that returns \texttt{TRUE} if the union of
\texttt{u} and \texttt{v} has 100 elements and \texttt{FALSE} otherwise.

\begin{Shaded}
\begin{Highlighting}[]
\KeywordTok{length}\NormalTok{(}\KeywordTok{union}\NormalTok{(u,v)) }\OperatorTok{>}\StringTok{ }\DecValTok{100}
\end{Highlighting}
\end{Shaded}

\begin{verbatim}
## [1] FALSE
\end{verbatim}

\hypertarget{question-32}{%
\subsection{Question 32}\label{question-32}}

\emph{(3 points)}

\emph{Notes 1B (5) and Notes 1D (7)}

Display the (sorted!) values of \texttt{u} that do not appear in
\texttt{v}.

\begin{Shaded}
\begin{Highlighting}[]
\KeywordTok{sort}\NormalTok{(}\KeywordTok{setdiff}\NormalTok{(u,v))}
\end{Highlighting}
\end{Shaded}

\begin{verbatim}
## [1] 1 3 9 10 19 20 24 34 38 39 44 55 60 63 70 72 75 82 85
## [20] 89 97 99 100
\end{verbatim}

\hypertarget{question-33}{%
\subsection{Question 33}\label{question-33}}

\emph{(4 points)}

\emph{Notes 1D (5-7)}

Display a table showing how many values that are in \texttt{v} but not
in \texttt{u} fall into the bins {[}1,50{]} and {[}51,100{]}.

\end{document}
